% arara: clean: {
% arara: --> extensions:
% arara: --> ['aux','log','nav',
% arara: --> 'out','snm','toc','pdf']
% arara: --> }
% arara: lualatex: {
% arara: --> interaction: batchmode
% arara: --> }
% arara: clean: {
% arara: --> extensions:
% arara: --> ['aux','log','nav',
% arara: --> 'out','snm','toc']
% arara: --> }
\PassOptionsToPackage{svgnames}{xcolor}
\documentclass[
	spanish,
	9pt,
	utf8,
	xcolor=table,
	handout,
	aspectratio=1610,
	professionalfonts,
	notheorems,
	mathserif,
	% t
]{beamer}

\usepackage[spanish,es-sloppy]{babel}
\usepackage{enumerate}
\usepackage{multicol}
\usepackage{array}
\usepackage[linesnumbered,ruled,boxed,vlined,spanish]{algorithm2e}
\usepackage{algorithmicx}

\newcolumntype{x}[1]{>{\centering\arraybackslash\hspace{0pt}}p{#1}}

\newcounter{savedenum}
\newcommand*{\saveenum}{\setcounter{savedenum}{\theenumi}}
\newcommand*{\resume}{\setcounter{enumi}{\thesavedenum}}

\setbeamertemplate{navigation symbols}{}
\setbeamertemplate{footline}{}
\setbeamertemplate{headline}{}

\begin{document}

\begin{frame}

	\begin{enumerate}
		\item
		      Escribir el número decimal correspondiente a los siguientes números.

		      \begin{multicols}{4}
			      \begin{enumerate}[a)]
				      \item
				            ${1101110}_{\left(2\right)}$.

				      \item
				            $1101110,01_{\left(2\right)}$.

				      \item
				            ${100111,101}_{\left(2\right)}$.

				      \item
				            ${101101,001}_{\left(2\right)}$.
			      \end{enumerate}
		      \end{multicols}

		\item
		      Escribir en base $2$ los siguientes números dados en base $10$.

		      \begin{multicols}{4}
			      \begin{enumerate}[a)]
				      \item
				            $2324,6$.

				      \item
				            $3475,52$.

				      \item
				            $45632$.

				      \item
				            $1234,83$.
			      \end{enumerate}
		      \end{multicols}

		\item
		      Si tenemos $\beta=10$, $N=11$ y $k=6$.
		      Entonces, disponemos de $k=6$ dígitos para la parte fraccionaria y $N-k-1$ dígitos para la parte entera.
		      Escribe la representación de los siguientes números:

		      \begin{multicols}{4}
			      \begin{enumerate}[a)]
				      \item
				            $38,214$.

				      \item
				            $40,9561$.

				      \item
				            $-0,000876$.

				      \item
				            $0,952$.
			      \end{enumerate}
		      \end{multicols}

		\item
		      Sea la longitud de palabra de $N=4$ bits, genere una tabla que muestre la representación decimal de los números $+7,+6,+5,+4,+3,+2,+1,+0,-0,-1,-2,-3,-4,-5,-6,-7$ y $-8$ en la representación de su signo-magnitud y complemento a dos.
		      Es decir,

		      \

		      \begin{table}[ht!]
			      \centering
			      \begin{tabular}{|x{2.5cm}|x{2.5cm}|x{3.1cm}|}
				      \hline
				      Representación decimal & Representación signo-magnitud & Representación complemento a dos \\
				      \hline$+7$             & $0111$                        & $0111$                           \\
				      \hline
			      \end{tabular}
		      \end{table}

		      \saveenum
	\end{enumerate}
\end{frame}

\begin{frame}
	\begin{enumerate}
		\resume

		\item
		      Realice las operaciones aritméticas con enteros complemento a dos con una longitud de palabra de $N=4$ bits para las siguientes operaciones:

		      \begin{multicols}{4}
			      \begin{enumerate}[a)]
				      \item
				            $1001+0101$.

				      \item
				            $1100+0100$.

				      \item
				            $0011+0100$.

				      \item
				            $1100+1111$.
			      \end{enumerate}
		      \end{multicols}

		\item
		      Usando la idea de ítem anterior donde se produce desbordamiento entero de:
		      \begin{multicols}{2}
			      \begin{enumerate}[a)]
				      \item
				            $0101+0100$.

				      \item
				            $1001+1010$.
			      \end{enumerate}
		      \end{multicols}

		\item
		      Sean $a=0,000063381158$, $b=73,688329$ y $c=-73,687711$. Determine la aritmética de punto flotante para:
		      \begin{multicols}{3}
			      \begin{enumerate}[a)]
				      \item
				            $a+(b+c)$.

				      \item
				            $(a+b)+c$.

				      \item
				            $a+b+c$.
			      \end{enumerate}
		      \end{multicols}

		\item
		      Si tenemos $\beta=2, t=3, L=-2$ y $U=2$.
		      \begin{enumerate}[a)]
			      \item
			            Determine el intervalo donde se representa los números reales.

			      \item
			            Determine la cantidad de números reales que tiene dicho intervalo.

			      \item
			            Determine los números de máquina que contiene dicho intervalo.
		      \end{enumerate}

		\item
		      Si tenemos $\beta=2$, $t=3$, $L=-1$ y $U=2$.
		      \begin{enumerate}[a)]
			      \item
			            Determine el intervalo donde se representa los números reales.

			      \item
			            Determine la cantidad de números reales que tiene dicho intervalo.

			      \item
			            Determine los números de máquina que contiene dicho intervalo.
		      \end{enumerate}

		      \saveenum
	\end{enumerate}
\end{frame}

\begin{frame}
	\begin{enumerate}
		\resume

		\item
		      Usando el ítem 8 parte c), determine:

		      \begin{multicols}{4}
			      \begin{enumerate}[a)]
				      \item
				            $\dfrac{24}{32}\oplus\dfrac{7}{32}$.

				      \item
				            $\dfrac{24}{32}\ominus\dfrac{7}{32}$.

				      \item
				            $\dfrac{24}{32}\otimes\dfrac{7}{32}$.

				      \item
				            $\dfrac{24}{32}\oslash\dfrac{7}{32}$.
			      \end{enumerate}
		      \end{multicols}

		\item
		      Dada la sucesión definida por:

		      \begin{equation*}
			      u_{0}=3/2,\quad u_{1}=5/3,\quad
			      \text { y }u_{n+1}=2003-\frac{6002}{u_{n}}+\frac{4000}{u_{n} u_{n-1}}.
		      \end{equation*}

		      \begin{multicols}{2}
			      \begin{enumerate}[a)]
				      \item
				            Demuestre que dicha sucesión converge hacia $2$.

				      \item
				            Evalue en \texttt{python} los $10$ primeros dígitos.
			      \end{enumerate}
		      \end{multicols}

		\item
		      Tenemos el siguiente algoritmo, donde $A$ y $B$ se definen por:

		      \begin{algorithm}[H]
			      $A\leftarrow 1.0$\;
			      $B\leftarrow 1.0$\;
			      $p\leftarrow 0$\;
			      \Mientras{\normalfont $\left(\left(A+1\right)-A\right)-1=0$}{
				      $A\leftarrow 2\ast A$\;
				      $p\leftarrow p+1$\;
				      \Mientras{\normalfont $\left(\left(A+B\right)-A\right)-B\neq0$}{
					      $B\leftarrow B+1$\;
				      }
			      }
		      \end{algorithm}

		      Interprete los valores de $B$ y $p$ retornados en su máquina y en otra máquina.

		      \saveenum
	\end{enumerate}
\end{frame}

\begin{frame}
	\begin{enumerate}
		\resume

		\item
		      Realizar lo siguiente en \texttt{python}:

		      \begin{multicols}{2}
			      \begin{enumerate}[a)]
				      \item
				            $\left(1,2-1\right)-0,2$.

				      \item
				            $1,2-\left(1+0,2\right)$.
			      \end{enumerate}
		      \end{multicols}

		      Podemos concluir que:

		      \begin{multicols}{2}
			      \begin{enumerate}[a)]
				      \item
				            La propiedad asociativa en el conjunto $\mathbb{F}$, ¿se cumple?
				      \item
				            Explique claramente el motivo de esto.
			      \end{enumerate}
		      \end{multicols}

		\item
		      Escriba una función en \texttt{python} de la fórmula de la ecuación de segundo grado:
		      \begin{equation*}
			      ax^{2}+bx+c=0.
		      \end{equation*}

		      \begin{enumerate}[a)]
			      \item
			            Determine la solución de $x^{2}+10^{8} x+1=0$, enseguida evalúe dicho número en el polinomio llamado residuo, e interprete el resultado.
			      \item
			            Encontrar una expresión matemáticamente equivalente que nos da una solución con menor residuo que el anterior.
		      \end{enumerate}

		\item
		      En \texttt{python} dar los valores de las funciones:

		      \begin{align*}
			      f\left(x\right) & =\sqrt{x^{2}+1}-1               \\
			      g\left(x\right) & =\frac{x^{2}}{\sqrt{x^{2}+1}+1}
		      \end{align*}

		      para una sucesión de valores de $x$ como $8^{-1}$, $8^{-2}$, $8^{-3}$, \ldots.
		      ¿Los resultados son iguales?

		      \saveenum
	\end{enumerate}
\end{frame}

\begin{frame}
	\begin{enumerate}
		\resume

		\item
		      Si $x_{0}>-1$ entonces la sucesión

		      \begin{equation*}
			      x_{n+1}=
			      2^{n+1}\left[\sqrt{1+2^{-n}x_{n}}-1\right]
		      \end{equation*}

		      converge hacia $\ln\left(x_{0}+1\right)$.
		      Modifique esta fórmula de tal manera que no haya pérdida de dígitos significativos.

		\item
		      En clase se ha visto distintas formas de sumar:

		      \begin{multicols}{2}
			      \begin{enumerate}[a)]
				      \item
				            \texttt{sum([0,1]*10)}.

				      \item
				            \texttt{math.fsum([0,1]*10)}.
			      \end{enumerate}
		      \end{multicols}

		      Explique:

		      \begin{multicols}{2}
			      \begin{enumerate}[a)]
				      \item La diferencia de ambos métodos y la diferencia principal.
				      \item Explique el porqué en el primero no se obtiene el valor de $1$.
			      \end{enumerate}
		      \end{multicols}

		\item
		      Muestre ejemplos que $\operatorname{fl}\left(\operatorname{fl}\left(xy\right)z\right)\neq\operatorname{fl}\left(x\operatorname{fl}\left(yz\right)\right)$.

		\item
		      Demuestre que $\frac{4}{5}$ no se puede representar de manera exacta como número de máquina.
		      ¿Cuál será el número de máquina más cercano?
		      ¿Cuál será el error de redondeo relativo que se produce
		      cuando se almacena internamente este número?

		\item
		      Del desarrollo en serie de:

		      \begin{equation*}
			      e^{x}=
			      \sum_{n=1}^{+\infty}
			      \frac{x^{n}}{n!}
		      \end{equation*}

		      Determine el error minimo, y el número de términos necesarios hasta llegar a dicho error mínimo.

	\end{enumerate}
\end{frame}

\end{document}