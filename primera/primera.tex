% arara: clean: {
% arara: --> extensions:
% arara: --> ['aux','log','nav',
% arara: --> 'out','snm','toc','pdf']
% arara: --> }
% arara: lualatex: {
% arara: --> interaction: batchmode
% arara: --> }
% arara: clean: {
% arara: --> extensions:
% arara: --> ['aux','log','nav',
% arara: --> 'out','snm','toc']
% arara: --> }
\documentclass{beamer}

\usepackage{enumerate}
\usepackage{multicol}
\usepackage{array}

\newcolumntype{x}[1]{>{\centering\arraybackslash\hspace{0pt}}p{#1}}

\newcounter{savedenum}
\newcommand*{\saveenum}{\setcounter{savedenum}{\theenumi}}
\newcommand*{\resume}{\setcounter{enumi}{\thesavedenum}}

\begin{document}

\begin{frame}

	\begin{enumerate}
		\item Escribir el número decimal correspondiente a los siguientes números
		      \begin{multicols}{2}
			      \begin{enumerate}[a)]
				      \item ${1101110}_{2}$.
				      \item $1101110,01_{2}$.
				      \item ${100111,101}_{2}$.
				      \item ${101101,001}_{2}$.
			      \end{enumerate}
		      \end{multicols}
		\item Escribir en base $2$ los siguientes números dados en base $10$.
		      \begin{multicols}{2}
			      \begin{enumerate}[a)]
				      \item $2324,6$.
				      \item $3475,52$.
				      \item $45632$.
				      \item $1234,83$.
			      \end{enumerate}
		      \end{multicols}
		\item Si tenemos $\beta=10$, $N=11$ y $k=6$. Entonces, disponemos de $k=6$ dígitos para la parte fraccionaria y $N-k-1$ dígitos para la parte entera. Escribe la representación de los siguientes números:
		      \begin{multicols}{2}
			      \begin{enumerate}[a)]
				      \item $38,214$.
				      \item $40,9561$.
				      \item $-0,000876$.
				      \item $0,952$.
			      \end{enumerate}
		      \end{multicols}
		      \saveenum
	\end{enumerate}
\end{frame}

\begin{frame}
	\begin{enumerate}
		\resume
		\item Sea la longitud de palabra de $N=4$ bits, genere una tabla que muestre la representación decimal de los números $+7,+6,+5,+4,+3,+2,+1,+0,-0,-1,-2,-3,-4,-5,-6,-7$ y $-8$ en la representación de su signo-magnitud y complemento a dos. Es decir
		      \begin{table}[ht!]
			      \centering
			      \begin{tabular}{|x{2.5cm}|x{2.5cm}|x{3.1cm}|}
				      \hline
				      Representación decimal & Representación signo-magnitud & Representación complemento a dos \\
				      \hline$+7$             & $0111$                        & $0111$                           \\
				      \hline
			      \end{tabular}
		      \end{table}
		\item Realice las operaciones aritmética con enteros complemento a dos con una longitud de palabra de $N=4$ bits para las siguientes operaciones:
		      \begin{multicols}{2}
			      \begin{enumerate}[a)]
				      \item $1001+0101$.
				      \item $1100+0100$.
				      \item $0011+0100$.
				      \item $1100+1111$
			      \end{enumerate}
		      \end{multicols}
		      \saveenum
	\end{enumerate}
\end{frame}

\begin{frame}
	\begin{enumerate}
		\resume
		\item Usando la idea de ítem anterior donde se produce desbordamiento entero de:
		      \begin{multicols}{2}
			      \begin{enumerate}[a)]
				      \item $0101+0100$.
				      \item $1001+1010$.
			      \end{enumerate}
		      \end{multicols}
		\item Sean $a=0,000063381158$, $b=73,688329$ y $c=-73,687711$. Determine la aritmética de punto flotante para:
		      \begin{multicols}{3}
			      \begin{enumerate}[a)]
				      \item $a+(b+c)$.
				      \item $(a+b)+c$.
				      \item $a+b+c$.
			      \end{enumerate}
		      \end{multicols}
		\item Si tenemos $\beta=2, t=3, L=-2$ y $U=2$.
		      \begin{enumerate}[a)]
			      \item Determine el intervalo donde se representa los números reales.
			      \item Determine la cantidad de números reales que tiene dicho intervalo.
			      \item Determine los números de máquina que contiene dicho intervalo.
		      \end{enumerate}
		      \saveenum
	\end{enumerate}
\end{frame}

\begin{frame}
	\begin{enumerate}
		\resume
		\item Si tenemos $\beta=2$, $t=3$, $L=-1$ y $U=2$.
		      \begin{enumerate}[a)]
			      \item Determine el intervalo donde se representa los números reales.
			      \item Determine la cantidad de números reales que tiene dicho intervalo.
			      \item Determine los números de máquina que contiene dicho intervalo.
		      \end{enumerate}
		\item Usando el ítem 8 parte c), determine:
		      \begin{multicols}{4}
			      \begin{enumerate}[a)]
				      \item $\dfrac{24}{32}\oplus\dfrac{7}{32}$.
				      \item $\dfrac{24}{32}\ominus\dfrac{7}{32}$.
				      \item $\dfrac{24}{32}\otimes\dfrac{7}{32}$.
				      \item $\dfrac{24}{32}\oslash\dfrac{7}{32}$.
			      \end{enumerate}
		      \end{multicols}
	\end{enumerate}
\end{frame}

\end{document}