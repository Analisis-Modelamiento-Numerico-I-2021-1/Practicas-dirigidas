% arara: clean: {
% arara: --> extensions:
% arara: --> ['aux','log','nav',
% arara: --> 'out','snm','toc','pdf']
% arara: --> }
% arara: lualatex: {
% arara: --> interaction: batchmode
% arara: --> }
% arara: clean: {
% arara: --> extensions:
% arara: --> ['aux','log','nav',
% arara: --> 'out','snm','toc']
% arara: --> }
\PassOptionsToPackage{svgnames}{xcolor}
\documentclass[
	spanish,
	9pt,
	utf8,
	xcolor=table,
	handout,
	aspectratio=1610,
	professionalfonts,
	notheorems,
	mathserif,
	% t
]{beamer}

\usepackage[spanish,es-sloppy]{babel}
\usepackage{enumerate}
\usepackage{multicol}
\usepackage{array}

\newcounter{savedenum}
\newcommand*{\saveenum}{\setcounter{savedenum}{\theenumi}}
\newcommand*{\resume}{\setcounter{enumi}{\thesavedenum}}

\setbeamertemplate{navigation symbols}{}
\setbeamertemplate{footline}{}
\setbeamertemplate{headline}{}

\begin{document}

\begin{frame}

	\begin{enumerate}
		\item
		      Determine la notación de Landau de las siguientes funciones.

		      \begin{multicols}{4}
			      \begin{enumerate}[a)]
				      \item

				            $\dfrac{1}{n^{2}}$.

				      \item

				            $\cos\left(n\right)$.

				      \item

				            $\sin\left(\dfrac{x}{n}\right)$.

				      \item

				            $\sqrt{n+1}-\sqrt{n}$.
			      \end{enumerate}
		      \end{multicols}

		\item
		      La pérdida de cifras significativas se puede evitar reordenando los cálculos.
		      Determine en los siguientes casos una forma equivalente que evite la pérdida de cifras significativas para valores indicados de $x$.

		      \begin{multicols}{4}
			      \begin{enumerate}[a)]
				      \item

				            $\ln\left(x+1\right)-\ln\left(x\right)$.

				      \item

				            $\sqrt{x^{2}+1}-x$.

				      \item
				            $1-\cos\left(x\right)$.

				      \item

				            $\sin\left(x\right)-x$.
			      \end{enumerate}
		      \end{multicols}

		\item
		      Sea $f\colon\mathbb{R}^{n}\to\mathbb{R}$ definida por

		      \begin{multicols}{3}
			      \begin{enumerate}[a)]
				      \item

				            $f\left(x\right)=\dfrac{x}{4}$, $n=1$.

				      \item

				            $f\left(x\right)=\sqrt{x}$, $n=1$.

				      \item

				            $f\left(x_{1},x_{2}\right)=x_{1}\cdot x_{2}$, $n=2$.
			      \end{enumerate}
		      \end{multicols}

		      Determine el número de condición.

		\item
		      En un aparcamiento hay $55$ vehículos entre coches y motos.
		      Si el total de ruedas es de $170$.
		      Determine

		      \begin{multicols}{2}
			      \begin{enumerate}[a)]
				      \item
				            Modele el problema.

				      \item
				            Determine la norma matricial de $A$.



				      \item
				            Determine el número de condicionamiento de $A$.

				      \item
				            Indique si está bien o mal condicionado.
			      \end{enumerate}
		      \end{multicols}

		      \

		      \saveenum
	\end{enumerate}
\end{frame}

\begin{frame}
	\begin{enumerate}
		\resume

		\item
		      Un fabricante de bombillas gana $0,3$ dólares por cada bombilla que sale de la fábrica, pero pierde
		      $0,4$ dolares por cada una que sale defectuosa.
		      Un día en el que fabricó $2100$ bombillas obtuvo
		      un beneficio de $484,4$ dólares.
		      Determine el número de bombillas buenas y defectuosa según el requerimiento siguiente.

		      \begin{multicols}{2}
			      \begin{enumerate}[a)]
				      \item

				            Modele el problema.

				      \item

				            Determine la norma matricial de $A$.

				      \item

				            Determine el número de condicionamiento de $A$.

				      \item

				            Indique si está bien o mal condicionado.
			      \end{enumerate}
		      \end{multicols}

		\item
		      Sean dos números tales que la suma de un tercio del primero más un quinto del segundo sea igual
		      a $13$ y que si se multiplica el primero por $5$ y el segundo por $7$ se obtiene $247$ como suma de los dos productos.
		      Determine los números según el requerimiento siguiente.

		      \begin{multicols}{2}
			      \begin{enumerate}[a)]
				      \item

				            Modele el problema.

				      \item

				            Determine la norma matricial de $A$ y $A^{-1}$.

				      \item

				            Determine el condicionamiento de $A$.

				      \item

				            Resolver el sistema usando eliminación de Gauß.
			      \end{enumerate}
		      \end{multicols}

		\item
		      El perímetro de un rectángulo es $64$ cm y la diferencia entre las medidas de la base y la altura es $6$ cm.
		      Determine las dimensiones de dicho rectángulo según el requerimiento siguiente.

		      \begin{multicols}{2}
			      \begin{enumerate}[a)]
				      \item

				            Modele el problema.

				      \item

				            Determine la norma matricial de $A$ y $A^{-1}$.

				      \item

				            Determine el condicionamiento de $A$.

				      \item

				            Resolver el sistema usando eliminación de Gauß.
			      \end{enumerate}
		      \end{multicols}

		      \saveenum
	\end{enumerate}
\end{frame}

\begin{frame}
	\begin{enumerate}
		\resume

		\item
		      Dos kilos de plátanos y tres de peras cuestan $8,80$ soles.
		      Cinco kilos de plátanos y cuatro de peras
		      cuestan $16,40$ soles.
		      Determine el costo de kilo del plátano y de la pera según el requerimiento siguiente.

		      \begin{multicols}{2}
			      \begin{enumerate}[a)]
				      \item

				            Modele el problema.

				      \item

				            Determine la norma matricial de $A$ y $A^{-1}$.

				      \item

				            Determine el condicionamiento de $A$.

				      \item

				            Resolver el sistema usando eliminación de Gauß con pivoteo.
			      \end{enumerate}
		      \end{multicols}

		\item
		      La edad de Manuel es el doble de la edad de su hija Ana.
		      Hace diez años, la suma de las edades de
		      ambos era igual a la edad actual de Manuel.
		      Determine la edad de ambos según el requerimiento siguiente.
		      \begin{multicols}{2}
			      \begin{enumerate}[a)]
				      \item

				            Modele el problema.

				      \item

				            Determine la norma matricial de $A$ y $A^{-1}$.

				      \item

				            Determine el condicionamiento de $A$.

				      \item

				            Resolver el sistema usando eliminación de Gauß-Jordan.
			      \end{enumerate}
		      \end{multicols}

		\item
		      José dice a Eva: mi colección de discos compactos es mejor que la tuya ya que si te cedo $10$ tendríamos
		      la misma cantidad.
		      Eva le responde: reconozco que tienes razón.
		      Solo te faltan $10$ para doblarme en número.
		      Determine la cantidad de discos que tiene cada uno según el requerimiento siguiente.

		      \begin{multicols}{2}
			      \begin{enumerate}[a)]
				      \item

				            Modele el problema.

				      \item

				            Determine la norma matricial de $A$ y $A^{-1}$.

				      \item

				            Determine el condicionamiento de $A$.

				      \item

				            Resolver el sistema usando eliminación de Gauß-Jordan.
			      \end{enumerate}
		      \end{multicols}

		      \saveenum
	\end{enumerate}
\end{frame}

\begin{frame}
	\begin{enumerate}
		\resume

		\item

		      Dada la sucesión
		      \begin{math}
			      u_{n+1}=
			      2003-
			      \dfrac{6002}{u_{n}}+
			      \dfrac{4000}{u_{n}u_{n-1}}
		      \end{math}.

		      \begin{enumerate}[a)]
			      \item

			            Opcional Muestre, brevemente, que la solución general es de la forma
			            \begin{math}
				            v_{n}=
				            \dfrac{
					            \alpha+
					            \beta 2^{n+1}+
					            \gamma 2000^{n+1}
				            }{
					            \alpha+
					            \beta 2^{n}+
					            \gamma 2^{n}}
			            \end{math}

			      \item

			            Verifique que $\gamma=0$ para $u_{0}=\dfrac{3}{2}$ y $u_{1}=\dfrac{5}{3}$.

			      \item

			            Realice su código en \texttt{python} y verifique que $u_{n}$ converge computacionalmente hacia 2000, verifique si esto es correcto con el límite teórico.

			      \item

			            Analice la propagación de errores y clasifique el tipo de error.
		      \end{enumerate}

		\item
		      Dada la sucesión de Fibonacci $F_{n}$ definida por $F_{0}=F_{1}=1 \mathrm{y}$ su regla $F_{n+1}=F_{n}+F_{n-1}$.

		      \begin{enumerate}[a)]
			      \item

			            Analice la propagación de errores y clasifique el tipo de error.
		      \end{enumerate}

		\item
		      Construir las siguientes matrices de $N\times N$

		      \begin{equation*}\small
			      A_{1}=
			      \begin{pmatrix}
				      2  & -1     &        &        & 0  \\
				      -1 & 2      & -1     &        & 0  \\
				         & \ddots & \ddots & \ddots &    \\
				         &        & -1     & 2      & -1 \\
				      0  &        &        & -1     & 2
			      \end{pmatrix},
			      A_{2}=
			      \begin{pmatrix}
				      16 / 3 & -8 / 3 & 0      &        & 0      &        &        & 0      \\
				      -8 / 3 & 14 / 3 & -8 / 3 & 1 / 3  & 0      &        &        &        \\
				      0      & -8 / 3 & 16 / 3 & -8 / 3 & 0      &        &        &        \\
				      0      & 1 / 3  & -8 / 3 & 14 / 3 & -8 / 3 & 1 / 3  & 0      &        \\
				             & \ddots & \ddots & \ddots & \ddots & \ddots & \ddots &        \\
				      0      &        &        & 0      & -8 / 3 & 16 / 3 & -8 / 3 & 0      \\
				      0      &        &        & 0      & 1 / 3  & -8 / 3 & 14 / 3 & -8 / 3 \\
				      0      &        &        &        &        & 0      & -8 / 3 & 16 / 3
			      \end{pmatrix},
		      \end{equation*}


		      \begin{enumerate}[a)]
			      \item

			            Determine el número de condición para $N=8$ para cada matriz.

			      \item

			            Determine el número de condición para $N=12$ para cada matriz.

			      \item

			            Determine el número de condición para $N=14$ para cada matriz.
		      \end{enumerate}

		      \saveenum
	\end{enumerate}
\end{frame}

\begin{frame}
	\begin{enumerate}
		\resume

		\item
		      Dada la ecuación $Ax=b$:

		      \begin{equation*}
			      \begin{pmatrix}
				      10 & 7 & 8  & 7  \\
				      7  & 5 & 6  & 5  \\
				      8  & 6 & 10 & 9  \\
				      7  & 5 & 9  & 10
			      \end{pmatrix}
			      \begin{pmatrix}
				      x_{1} \\
				      x_{2} \\
				      x_{3} \\
				      x_{4}
			      \end{pmatrix}=
			      \begin{pmatrix}
				      32 \\
				      23 \\
				      33 \\
				      31
			      \end{pmatrix}
		      \end{equation*}

		      \begin{enumerate}[a)]
			      \item

			            Determine la inversa de $A$, su determinante, y verifique si es simétrica y definida positiva, de
			            manera similar determine su solución.

			      \item

			            Resuelva la ecuación matricial perturbando $b^{T}=\left(32,122,933,130,9\right)$.

			      \item

			            Resuelva ahora con la perturbación
			            \begin{equation*}
				            \begin{pmatrix}
					            10   & 7    & 8,1  & 7,2  \\
					            7,08 & 5,04 & 6    & 5    \\
					            8    & 5,98 & 9,89 & 9    \\
					            6,99 & 4,99 & 9    & 9,98
				            \end{pmatrix}
				            \begin{pmatrix}
					            x_{1} \\
					            x_{2} \\
					            x_{3} \\
					            x_{4}
				            \end{pmatrix}=
				            \begin{pmatrix}
					            32 \\
					            23 \\
					            33 \\
					            31
				            \end{pmatrix}
			            \end{equation*}

			      \item

			            Explique claramente lo determinado en los ítemes anteriores.
		      \end{enumerate}

		\item

		      Sea $A$ una matriz cuadrada invertible y $\alpha\in\mathbb{R}$.
		      Demuestre:
		      \begin{math}
			      \kappa\left(A\right)=
			      \kappa\left(\alpha A\right)=
			      \kappa\left(A^{-1}\right)
		      \end{math}.

		\item
		      Denotamos por
		      \begin{math}
			      {\left\|A\right\|}_{2}=
			      \sqrt{\rho\left(A^{T}A\right)}
		      \end{math}
		      donde
		      \begin{math}
			      \rho\left(B\right)=
			      \max_{1\leq i\leq n}
			      \left|\lambda_{i}\left(B\right)\right|
		      \end{math}
		      y
		      \begin{math}
			      {\left(
				      \lambda_{i}\left(B\right)
				      \right)}_{
				      1\leq i\leq n
			      }
		      \end{math}
		      es el conjunto de los valores propios de $B$.
		      Con esta norma definimos el condicionamiento
		      \begin{math}
			      \kappa_{2}=
			      {\left\|A\right\|}_{2}{
			      \left\|A^{-1}\right\|
			      }_{2}
		      \end{math}.
		      Demostrar que:

		      \begin{enumerate}[a)]
			      \item

			            \begin{math}
				            \kappa_{2}
				            \left(A\right)=
				            \kappa_{2}
				            \left(A^{T}\right)
			            \end{math}.
		      \end{enumerate}

		      \saveenum
	\end{enumerate}
\end{frame}

\begin{frame}
	\begin{enumerate}
		\resume

		\item

		      Sean
		      \begin{math}
			      A=
			      \begin{pmatrix}
				      a & b \\ c & d
			      \end{pmatrix}
		      \end{math}
		      definida positiva y $\alpha,\beta\in\mathbb{R}^{+}$.
		      Denotamos
		      \begin{math}
			      \Delta=
			      \begin{pmatrix}
				      \alpha & 0 \\ 0 & \beta
			      \end{pmatrix}
		      \end{math},
		      definimos también
		      \begin{math}
			      D=
			      \begin{pmatrix}
				      a & 0 \\ 0 & b
			      \end{pmatrix}
		      \end{math}.
		      Mostrar que para cualquier elección de $\alpha$ y $\beta$ tenemos:

		      \begin{multicols}{3}
			      \begin{enumerate}[a)]
				      \item

				            \begin{math}
					            \kappa_{2}
					            \left(D^{-1}A\right)\leq
					            \kappa_{2}
					            \left(\Delta^{-1}A\right)
				            \end{math},

				      \item

				            \begin{math}
					            \kappa_{2}
					            \left(D^{-1}A\right)\leq
					            \kappa_{2}
					            \left(A\right)
				            \end{math},

				      \item

				            \begin{math}
					            \kappa_{2}
					            \left(D\right)\leq
					            \kappa_{2}
					            \left(A\right)
				            \end{math}.
			      \end{enumerate}
		      \end{multicols}

		\item
		      Sobre el método de Doolitle.

		      \begin{enumerate}[a)]

			      \item

			            Implementar dicho método en \texttt{python} llamado \texttt{doolitle}.

			      \item

			            Crear matrices $L_{n}$ y $U_{n}$ de orden $n\times n$ para $n=3,4,\dotsc,15$ como las mencionadas en el teorema $3$ del tema factorización $LU$ de la semana $3$, luego almacenar $A_{n}=L_{n}U_{n}$.

			      \item

			            Obtener la complejidad de dicho algoritmo de forma teórica.

			      \item

			            Del ítem b.
			            Obtener una gráfica de tiempo que toma el método \texttt{doolitle} contra el orden de la matriz $n$.
		      \end{enumerate}

		\item

		      Sobre el método de Crout.

		      \begin{enumerate}[a)]
			      \item

			            Implementar dicho método en \texttt{python} llamado \texttt{crout}.

			      \item

			            Crear matrices $L_{n}$ y $U_{n}$ de orden $n\times n$ para $n=3,4,\dotsc,15$ como las mencionadas en el teorema $2$ del tema factorización $LU$ de la semana $3$, luego almacenar $A_{n}=L_{n}U_{n}$.

			      \item

			            Obtener la complejidad de dicho algoritmo de forma teórica.

			      \item

			            Del item b.
			            Obtener una gráfica de tiempo que toma el método \texttt{crout} contra el orden de la matriz $n$.
		      \end{enumerate}s

		\item
		      Sobre el método de Cholesky.

		      \begin{enumerate}[a)]
			      \item

			            Implementar dicho método en \texttt{python} llamado \texttt{cholesky}.


			      \item

			            Crear una matriz $L_{n}$ de orden $n\times n$ para $n=3,4,\dotsc,15$ como las mencionadas en el teorema $6$ del tema factorización $LU$ de la semana $3$, luego almacenar $A_{n}=L_{n}L_{n}^{T}$.

			      \item

			            Obtener la complejidad de dicho algoritmo de forma teórica.

			      \item

			            Del ítem b.
			            Obtener una gráfica de tiempo que toma el método \text{cholesky} contra el orden de la matriz $n$.
		      \end{enumerate}

	\end{enumerate}
\end{frame}

\end{document}